% !TeX spellcheck = hu_HU
%----------------------------------------------------------------------------
\chapter{Üzemeltetés}
%----------------------------------------------------------------------------
Korábbi fejezetben írtam arról, hogy az üzemeltetést Heroku és Vercel segítségével oldottam meg.
Azonban ez az alkalmazás tényleges üzembe helyezése után ez nem feltétlenül a legköltséghatékonyabb módja az üzemeltetésének.
Ezért igyekeztem alternatívákat kínálni az esetleges üzembehelyezőknek.

\section{Docker}
Az üzemeltetés megkönnyítése érdekében minden komponenshez készítettem egy–egy dockerfile–t és docker-compose file–t.

A frontendhez tartozó Docker felépítése három lépésből áll. Első lépésben a szükséges függőségeket telepítjük a yarn csomagkezelő segítségével.
Második lépésben a TypeScript kódot fordítjuk JavaScript kódra, a harmadik lépésben pedig egyszerűen elindítjuk az alkalmazást.

A három lépéses konténer készítés oka a docker cachelésének maximális kihasználása.
Ez különösen nagy segítség lehet, ha ugyan azt a verziót több környezetbe is szeretnénk élesíteni.

\begin{lstlisting}[language=TeX, caption=Frontend Dockerfile]
# 1 - Install dependecies
FROM node:12-alpine AS dependencies

WORKDIR /app
COPY package.json yarn.lock ./
RUN yarn install

# 2 - Build
FROM node:12-alpine AS build

ENV NODE_ENV=production
WORKDIR /app
COPY . .
COPY --from=dependencies /app/node_modules ./node_modules
RUN yarn build

# 3 - Run
FROM node:12-alpine AS run

WORKDIR /app
ENV NODE_ENV=production
COPY --from=dependencies /app/node_modules ./node_modules
COPY --from=build /app/public ./public
COPY --from=build /app/next.config.js ./
COPY --from=build /app/.next ./.next
CMD ["node_modules/.bin/next", "start"]
\end{lstlisting}

A backendhez tartozó container sokkalta egyszerűbb. 
A hozzá tartozó docker-compose file tartalmazza a backend és az adatbázis elindítását, valamint a közöttük lévő kapcsolat kiépítését és a Prisma Studio indítását is.
A Prisma Studio és az alkalmazás konkurensen futnak egy containeren belül.
Fontos megjegyezni, hogy a Studiot csak feltétlenül szükséges esetben használjuk, érdemes tiltani a hozzáférést ugyanis nem tartalmaz semmilyen beépített jelszavas védelmet.

\begin{lstlisting}[language=TeX, caption=Backend docker compose]
version: '3'
services:
  server:
    build: .
    links:
      - postgres
    depends_on:
      - postgres
    ports:
      - '${SERVER_PORT}:4000'
      - '${STUDIO_PORT}:5555'
    environment:
      NODE_ENV: ${NODE_ENV}
      DATABASE_URL: ${DATABASE_URL}
      APP_SECRET: ${APP_SECRET}
    networks:
      - server-network
  postgres:
    image: postgres:12.5
    restart: always
    environment:
      POSTGRES_USER: prisma
      POSTGRES_PASSWORD: prisma
      POSTGRES_DB: prisma
    networks:
      server-network:
        aliases:
          - postgresql.db
    volumes:
      - db_data:/var/lib/postgresql/data
volumes:
  db_data:
networks:
  server-network:
\end{lstlisting}

%----------------------------------------------------------------------------
