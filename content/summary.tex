%----------------------------------------------------------------------------
\chapter{Összefoglalás}
%----------------------------------------------------------------------------

A fejlesztés során rengeteg olyan problémát kellett megoldanom, melyekkel egyébként nem találkoztam volna.
Elmélyültem olyan technologiákban, melyek napjainkban "divatosak" és piaci környezetben is megállják a helyüket.
Ezeket a későbbiekben alkalmazhatom, mint tanulányaimban, mint munkám során.
%----------------------------------------------------------------------------

\section{Tovább fejlesztési lehetőségek}
A félév során temérdek új fejlesztési lehetőség jutott eszembe, melyeket megvalósítva egy még jobb és méginkább a piaci igényeknek megfelelő alkalmazást kaphatunk.
A szakdolgozat keretein belül igyekeztem az elengedhetetlen funkciókat a lehető legjobban megvalósítani és inkább a fejlesztői eszköz tár felépítését tartottam fontosnak, mint a rengeteg funkció belezsúfolását egy alkalmazásba.
Ennek köszönhetően remélhetőleg egy hosszú távon is fejleszthető és fentartható alkalmazás születet.
A félév végeztével folytatnám a munkát, hogy a lehető legtöbb piaci igényt legyen képes kiszolgálni az alkalmazásom.
%----------------------------------------------------------------------------

\subsection{Keresés}
A jelenlegi keresés egy nagyon egyszerű string összehasonlításon alapul, már egyetlen karakter elgépelése esetén sem talál egyezést.
Ennek a problémának a megoldására több lehetőség is kinálkozik, ilyen például az Elastic Search vagy a PostgreSQL-be épített full-text search.
Ezeknek az alapvető működési elve, hogy nem magában az adatbázisban keres hanem létrehoz egy index-szelt szótárat és ezt hasonlítja a keresési kifejezéshez.
Ennek köszönhetően nagy adatbázisok esetén is gyors keresés érhető el.
%----------------------------------------------------------------------------

\subsection{Egyedi tulajdonságok kezelése}
Már a félév elején felvetödött a raktárban tárolt eszközök egyedi tulajdonságainak kezelése, mint ötlet.
A tervezés során kiderült, hogy ez jóval több időt venne igénybe, mint azt az elején gondoltam, így a megvalósítását kihagytam a szakdolgozatból.

A koncepció az volt, hogy kategoriákat hozhattunk volna létre minden raktáron belül.
A kategoriákhoz egy név megadása után felvehetőek lettek volna a tulajdonságok nevei és típusai.
Ezután az eszköz felvételekor kiválasztjuk, hogy milyen kategoriába, kategoriákba tartozik. 
Így a kategoriák révén már tudjuk azt, hogy milyen egyedi tulajdonságai lehetnek.
%----------------------------------------------------------------------------