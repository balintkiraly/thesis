%----------------------------------------------------------------------------
\chapter{Feladat specifikáció}
%----------------------------------------------------------------------------

A fejezet kitér az alkalmazással szemben támasztott funkcionális és nem funkcionális követelményekre.

%----------------------------------------------------------------------------
\section{Funkciónális követelmények}
%----------------------------------------------------------------------------

%----------------------------------------------------------------------------
\subsection{Regisztráció}
%----------------------------------------------------------------------------
Az elkészítendő alkalmazásban legyen lehetőség felhasználót létrehozni egy regisztrációs oldalon.
A regisztráció során a felhasználó nevét, email címét, jelszavát valamint a jelszavának megerősítését kérjük.
További fontos követelmény, hogy egy email címhez csak egy felhasználó tartozhat

%----------------------------------------------------------------------------
\subsection{Bejelentkezés}
%----------------------------------------------------------------------------
A regisztráció során megadott email cím és jelszó segítségével a látogatónak képesnek kell lennie authentikálni magát.
A rendszer csak authentikált felhasználók számára legyen elérhető.
Nem authentikált felhasználóknak csak a bejelentkezés és a regisztráció opciókat kínáljuk fel.

%----------------------------------------------------------------------------
\subsection{Raktár épületek kezelése}
%----------------------------------------------------------------------------
Az alkalmazással szemben követelmény, hogy képesnek kell lennie több raktár (raktár épület) kezelését.
Ez alatt értjük a raktár létrehozását, és szerkesztését valamint az ezekhez tartozó jogosultságok menedzselését.
A raktárról tároljuk a méreteit, a nevét és természetesen a szerkesztésre jogosult felhasználók listáját.

%----------------------------------------------------------------------------
\subsection{Tárolók kezelése}
%----------------------------------------------------------------------------
Minden raktárba tárolók helyezhetőek. A tárolókat a nevükkel és meretükkel egyűtt rögzíthetjuk.
Amennyiben a felhasználó rendelkezik a megfelelő jogosultsággal az adott raktáron belül, legyen lehetősége a tárolók szerkesztésre, létrehozására és törlésére.

%----------------------------------------------------------------------------
\subsection{Eszközök kezelése}
%----------------------------------------------------------------------------
A hierarchia harmadik szintjén helyezkednek el az eszközökök. 
Minden eszköz rendelkezzen az alábbi tulajdonságokkal.
Név, amely az egyszerű azonosítást és a kereshetőséget biztosítja.
Érték, ami az eszköz raktárba vételekori értékét tartalmazza.
Minden egyes leltárba vett eszköhöz legyen lehetőségünk a kiadások felvételére.
Minden kiadáshoz egy összeg és egy leírás tartozik, amely a kiadás okának magyarázatára szolgál.

%----------------------------------------------------------------------------
\subsection{Raktár térképes nézettel}
%----------------------------------------------------------------------------
A raktárakban a tárolók elhelyezkedését jelenítsük meg egy felülnézeti, térképes nézet formájában is.
A tárolók mozgatását nem szükséges megvalósítani ezen a térképen.
Ennek oka, hogy míg az eszközök poziciója gyakran változik a tárolók fixen telepítve vannak.
Ennek a funkciónak a nem implementálása felesleges félreértések elkerülését is szolgálja.

%----------------------------------------------------------------------------
\subsection{Tároló térképes nézettel}
%----------------------------------------------------------------------------
Minden tároló oldalán jelenítsünk meg egy képet a tároló tartalmával.
A tárolt eszközöket egyszerű téglalappal reprezentáljuk.
Fontos követelmény, hogy a ezen a nézeten legyen lehetőség az eszközök mozgatására is.
A mozgatást a felhasználó a mozgatni kívánt elemre kattintva, majd az egér ball billentyűjét lenyomva mozgathatja a tárolón belül.
A fent leírt módszer segítségével legyen lehetőség egy ideiglenes tárolóba rakni. 
Ezt az ideiglenes tárolót jelenítsük meg minden (az adott raktárban lévő) tároló oldalán, az eszközök tárolók közötti mozgatását megvalósítva.
%----------------------------------------------------------------------------

\subsection{Kereshetőség}
Az alkalmazásban legyen lehetőség a leltárba vett eszközök közötti keresésre.
Fontos, hogy a keresés segítségével ne csak az adott elemet kapjuk meg hanem annak az elhelyezkedését is könnyedén le tudja kérdezni a felhasználó.
%----------------------------------------------------------------------------

\subsection{Naplózás}
Egy alkalmazásnál fontos, hogy képesek legyünk nyomonkövetni az adatbázisba történt változások okait, különösen igaz ez egy raktár rendszer esetén.
A rendszer logoljon minden eszközzel kapcsolatos felhasználói interakciót, amely adatbázis művelethez vezet.
Tehát létrehozás, szerkesztés és törtlés. A megtekintések logolása nem szükséges.
%----------------------------------------------------------------------------

\section{Nem funkcionális követelmények}
%----------------------------------------------------------------------------

Az alkalmazással szemben természetesen nem csak funkciónális követelmények vannak.
A nem funkciónális követelmények is legalább annyira fontosak egy alkalmazás tervezésénél és fejlesztésénél, mint a funkciónális követelmények.

%----------------------------------------------------------------------------
\subsection{Webes párhuzamos működés}
%----------------------------------------------------------------------------

Az elkészülő programmal szemben támasztott követelmények közé tartozik az egyidejüleg több felhasználó kiszolgálása webböngésző segítségével.

%----------------------------------------------------------------------------
\subsection{Üzemeltetéssel szemben támasztott követelmények}
%----------------------------------------------------------------------------

Továbbá fontos követelmény, hogy az alkalmazás üzemeltetéséhez ne legyen szükség speciális hardware vagy speciális operációs rendszer.
