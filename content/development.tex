
%----------------------------------------------------------------------------
\chapter{Alkalmazás fejlesztése és működése}
%----------------------------------------------------------------------------

\section{Backend}
A backend architekurális felépítéséről a korábbi fejezetben volt szó. 
Most térjünk át az egyes rétegek és egyes funkciók megvalósításának bemutatására.
%----------------------------------------------------------------------------

\subsection{GraphQL Playground}
A legtöbb GraphQL server-hez lehetőségünk nyilik valamilyen interaktív GraphQL szerkesztő felület kiszolgálásra is.
Az alkalmazásban én a GraphQL Playground-ot használtam, a konfigurációt úgy valosítottam meg, hogy éles környezetbe ne szolgálja ki ezt a felületet, csak feljesztői környezet estén.
Ezt környezeti változók segítségével oldaottam meg.

Azon felül, hogy interaktívan, kódkiegészítéssel szerkeszhetjük a GraphQL kódunkat kapunk egy dokumentációt is, amely tartalmazza az elérhető Query-k és Mutation-ök listáját a lehetséges paraméterekkel és a válasz típusával együtt.

\begin{figure}[!ht]
  \centering
  \includegraphics[width=150mm, keepaspectratio]{figures/playground_docs.png}
  \caption{GraphQL Playground Docs}
  \label{fig:playgroundDocs}
\end{figure}
%----------------------------------------------------------------------------

\subsection{Modellek és kapcsolatok}
Az egyszerűség és teljesség kedvéért egy, az alkalmazásban nem létező modelen mutatom be a működést.

A model definiálásakor megadhatunk bármelyiket a sémában szereplő attributumok közül, azonban fontos figyelni arra, hogy a típusok megegyezzenek a sémában és a modellben.
A kapcsolatokat is egyszerű mező ként kezelhetjük, beállítható továbba az is, hogy a kapcsolat opcionális vagy kötelező.
Lehetőségünk van egyedi attributumok létrehozására is, amely az adot modell bármely attributumábaló származtaható, de tetszőleges kód futtatása is megengedett.

\begin{lstlisting}[style=ES6, caption={Példa model}]
import { objectType } from '@nexus/schema'

export const TypeOfModel = objectType({
  name: 'TypeOfModel',
  definition(t) {
    t.id('id')
    t.string('name')
    t.field('user', {
      type: 'User',
      nullable: false,
    }),
    t.field('custom', {
      type: 'User',
      nullable: false,
      resolve: ({id}) => `#${id}`
    })
  },
})
\end{lstlisting}
%----------------------------------------------------------------------------

\subsection{Resolver felépítése}
A Mutation-ökhöz és Query-khez tartozó resolverek minden esetben tartalmaznak egy függvényt, amely eldönti, hogy meghívásukkor milyen kódrészlet fusson le.
Ezen felül tartalmaz egy visszatérési típust, hogy pontosan tudjuk milyen típussal térhez vissza.
Opcionálisan tartalmazhat egy argumentum listát, ami meghatározza, hogy milyen paraméterek szükségesek és milyen paraméterek opcionálisak a meghívásához.
Ezen fellül ezeknek a paramétereknek a típusát is meghatározza. 
\begin{lstlisting}[style=ES6, caption={Eszköz létrehozás resolver}]
t.field('createItem', {
  type: 'Item',
  args: { data: CreateItemInput.asArg({ required: true }) },
  resolve: async (_, { data: { storage, ...rest } }, context: Context) => {
    const item = await context.prisma.item.create({
      data: {
        storage: { connect: storage },
        ...rest,
      },
    })
    await log({
      type: 'CREATE',
      entityId: item.id,
      entityName: 'Item',
      newValues: { storage, ...rest },
      context,
    })
    await publishItemEvent('itemCreated', item, context)
    return item
  },
})
\end{lstlisting}
%----------------------------------------------------------------------------

\subsection{Authentikáció és authorizáció}
Az authentikáció és authorizáció ellenörzésére a GraphQL Shield-et használtam. 
Ennek segítségével egy egyszerű JavaScript object-tel megadható, hogy egyes Query-k és Mutation-ök esetén, milyen validáció fusson le.
Lehetőséget biztosít egy úgynevezett fallback rule bállítására is, amely minden külön nem specifikált kérésnél fut le.
Ezzel valosítottam meg a bejelentkezett felhasználó validását.

\begin{lstlisting}[style=ES6, caption={GraphQL Shield}]
export const shield = GQLShield(
   {
     Query: {
       warehouses: isGlobalAdmin,
       logs: isGlobalAdmin,
     },
     Mutation: {
       login: allow,
       register: allow,
     },
   },
   {
     allowExternalErrors: true,
     fallbackRule: fallbackRule,
   },
)
\end{lstlisting}

Az authentikációt JSON Web Token segítségével végzem. 
Sikeres bejelentkezés esetén a backend egy tokent küld a frontend részére, melyet a böngészőbe elmentve a későbbiekben minden kéréshez csatolni tudunk.

\begin{figure}[!ht]
  \centering
  \includegraphics[width=150mm, keepaspectratio]{figures/login.png}
  \caption{JWT Bejelentkezés}
  \label{fig:JWT}
\end{figure}

%----------------------------------------------------------------------------

\section{Frontend}

%----------------------------------------------------------------------------

\subsection{Útvonalválasztás}
NextJS használatával alapértelmezetten fájl alapú útvonalválasztást használhatunk.
Ez azt jelenti, hogy az oldal URL-jét a fájl neve és a szülő mappák nevei határozzák meg.
Lehetőségünk van paraméterek használatára is, ezt szögletes zárójelek közé írt névvel jelezhetjük.
A paramétert ezzel a névvel fogjuk elérni a kódbázison belül is.
Ilyen paraméterrel jelzett adhatunk bármelyik szinten lévő fájlnak, de akár mappának is.

\begin{figure}[!ht]
  \centering
  \includegraphics[width=55mm, keepaspectratio]{figures/next_routing.png}
  \caption{Next mappa szerkezeten alapulú routing}
  \label{fig:NextRouting}
\end{figure}

%----------------------------------------------------------------------------

\subsection{Kódgenerálás}
%----------------------------------------------------------------------------

Ahogy az korábban már többször is szóba került az Apollo-nak hála remek kódgenerálási lehetőségeink vannak.
Alább egy példa keretein belül szeretném bemutatni ennek használatát

A GraphQL Mutation-t paraméterekkel ellátva szükséges megírnunk, hogy a kód generátor tudja lehetséges paramétereket. 
\begin{lstlisting}[style=ES6, caption={GraphQL Shield}]
mutation Login($email: String!, $password: String!) {
  login(data: { email: $email, password: $password }) {
    token
  }
}
\end{lstlisting}

A generálást futtatva azonnal használhatjuk az elkészült hook-okat (jelen esetben a useLoginMutation hook-ot).

A hálozati kapcsolat kezelésén felül megkapjuk annak az állapotát is, így tudjuk a felhasználói felület kinézetét a kérés állapotához kötni.
Például töltés esetén egy töltő képernyőt megjelenítani, vagy a hibákat kezelni.

\begin{lstlisting}[style=ES6, caption={Bejelentkezés kódrészlet}]
const router = useRouter()
const [login, { loading }] = useLoginMutation()
const toast = useToast()
const { setAuthToken } = useAuthToken()

const onSubmit = async (inputData) => {
  try {
    const {
      data: {
        login: { token },
      },
    } = await login({ variables: inputData })
    setAuthToken(token)
    router.push('/')
  } catch (error) {
    toast({
      title: error.message,
      status: 'error',
      duration: 3000,
      isClosable: true,
    })
  }
}
\end{lstlisting}
%----------------------------------------------------------------------------

\subsection{Validáció}
A felhasználótól érkező adatok minden esetben validálva vannak, ehhez a yup csomagot használtam.
A yup segítségével definiálhatunk egy sémát, amelyre a bemenetnek illeszkedni kell. 
Hiba esetén egy hibaüzenetet is generál az egyes mezőkhöz, így visszacsatolást adhatunk a felhasználónak, hogy melyik adat hibás és miért.
Amennyiben szeretnék eltérni az alapértelmezett beállításoktól, a hibaüzeneteket szövegezését a séma definiálással együtt adhatjuk meg.

\begin{lstlisting}[style=ES6, caption={Esköz validációs séma}]
import * as yup from 'yup'

export const itemSchema = yup.object().shape({
  name: yup.string().required(),
  image: yup.string().required(),
  value: yup.number().required().typeError('Must be a number'),
  positionX: yup.number().required().typeError('Must be a number'),
  positionY: yup.number().required().typeError('Must be a number'),
  positionZ: yup.number().required().typeError('Must be a number'),
  sizeX: yup.number().required().typeError('Must be a number'),
  sizeY: yup.number().required().typeError('Must be a number'),
  sizeZ: yup.number().required().typeError('Must be a number'),
})
\end{lstlisting}

A formok elküldését egy React hook-kal valósítottam meg.
A hook visszaadja a hibákat és lehetőséget biztosít a form újrabeállítására is.
Minden beviteli mezőt regisztrálni kell a form hook-ba, így biztosítva azok elérését.

\begin{lstlisting}[style=ES6, caption={Regisztrációnál használt form hook}]
const { register, handleSubmit, reset, errors } = useForm<Inputs>({
  resolver: yupResolver(itemSchema),
})
\end{lstlisting}

A hibákat egy JavaScript objektumban kapjuk meg, melynek a kulcsa minden esetben az adott beviteli mező neve.

\begin{lstlisting}[style=ES6, caption={Form}]
<form onSubmit={handleSubmit(onSubmit)}>
  <FormControl mb={4} isInvalid={!!errors.name}>
    <FormLabel htmlFor="name">Name</FormLabel>
    <Input name="name" type="text" ref={register} />
    <FormErrorMessage>{errors.name?.message}</FormErrorMessage>
  </FormControl>
  ...
</form>
\end{lstlisting}

%----------------------------------------------------------------------------

\subsection{Felhasználói felület}
A felhasználói felületet a ChakraUI könyvtár segítségével készítettem el.
A Chakra hála az összes általános komponens egy szép és minden lehetséges állapotra felkészített változatban rendelkezésemre állt.

\begin{figure}[!ht]
  \centering
  \includegraphics[width=150mm, keepaspectratio]{figures/reg.png}
  \caption{Regisztrációs oldal}
  \label{fig:reg}
\end{figure}

A regisztrációs oldalon (\refstruc{fig:reg}) láthatjuk a beiteli mezőket kölünböző állapotban.
Hibásan kitöltött mező esetén felhasználó azonnali visszajelzést kap a hibás adatról.
Javítás után a hibaüzenet automatikusan eltűník, amint megfelelő formátumú adatot gépelt be a felhasználó.

\begin{figure}[!ht]
  \centering
  \includegraphics[width=150mm, keepaspectratio]{figures/search.png}
  \caption{Keresés}
  \label{fig:search}
\end{figure}

A keresés eredményét egy 3 oszlopos táblázatban (\refstruc{fig:search}) jelenítjük meg.
A három oszlop segítségével egyértelműen meghatározható a keresett eszköz holléte.
Lehetőségünk van a keresett eszköz raktárára, tárolójára vagy magára az eszközre navigálnunk.

A tároló nézetén (\refstruc{fig:storage}) a kurzort valamelyik eszköz fölé mozgatva megjelenítjük annak a nevét és ezen felül kiemeljük a listában is, hogy elősegítsük az azonosítást.
Természetesen a másik irányba is megvalósul a kiemelés, tehát ha a listában választjük ki, akkor a térképes nézeten kiemelten fogjuk látni az éppen kiválasztott eszközt.

\begin{figure}[!ht]
  \centering
  \includegraphics[width=150mm, keepaspectratio]{figures/storage.png}
  \caption{Tároló oldal}
  \label{fig:storage}
\end{figure}

A raktár nézetén belül a tárolóhoz hasonlóan kiemeléssel jelezzük az éppen kijelölt tárolót.
A tárolókrol extra információként megjelenítjük az adott tároló becsült kihasználását.
A becslés a tároló méretéből és a benne tárolt eszközök méretéből számolt értkét.
Természetesen pontosan kihasználtságot nem tudunk adni, mivel a tárolókat általában nem lehet 100\%-osan kitölteni.

\begin{figure}[!ht]
  \centering
  \includegraphics[width=150mm, keepaspectratio]{figures/dark_mode.png}
  \caption{Dark mode}
  \label{fig:darkMode}
\end{figure}
A ChakraUI segítségével egyszerűen megvalósítható a sötét és világos téma is az alkalmazáshoz, valamint az e kettő közötti váltás is.
Ez jelen esetben egy sötétkék árnyalatú designt (\refstruc{fig:darkMode}) eredményezett. 
A bejelentkezett felhasználó az oldal alján található gomb segítségével válathat témát.
A válaszott témát az alkalmazás automatiksan elmenti a böngésző helyi tárolójába, így az oldal újboli meglátogatásakor már a korábban kiválasztott téma lesz érvényes.


%----------------------------------------------------------------------------
