
%----------------------------------------------------------------------------
\chapter{Alkalmazás fejlesztése és működése}
%----------------------------------------------------------------------------

\section{Backend}
A backend architekurális felépítéséről a korábbi fejezetben volt szó. 
Most térjünk át az egyes rétegek és egyes funkciók megvalósításának bemutatására.
%----------------------------------------------------------------------------

\subsection{GraphQL Playground}
A legtöbb GraphQL server-hez lehetőségünk nyilik valamilyen interaktív GraphQL szerkesztő felület kiszolgálásra is.
Az alkalmazásban én a GraphQL Playground-ot használtam, a konfigurációt úgy valosítottam meg, hogy éles környezetbe ne szolgálja ki ezt a felületet, csak feljesztői környezet estén.
Ezt környezeti változók segítségével oldaottam meg.

GraphQL Playground es dokumentáció
%----------------------------------------------------------------------------

\subsection{Modellek és kapcsolatok}
modellek és kapcsolatok
%----------------------------------------------------------------------------

\subsection{Resolver felépítése}

resolver felépítése
%----------------------------------------------------------------------------

\subsection{Authentikáció és authorizáció}
GraphQL Shield és JWT

shield szabályok

JWT Ábra, token kezelés, kód magyarázat


%----------------------------------------------------------------------------

\section{Frontend}

%----------------------------------------------------------------------------

\subsection{Útvonalválasztás}
NextJS routing mappa szerkezet lista/kép
%----------------------------------------------------------------------------

\subsection{Kódgenerálás}
%----------------------------------------------------------------------------

Apollo codegen

hook példa, és használata

%----------------------------------------------------------------------------

\subsection{Validáció}

Yup validációs séma és react form hook
%----------------------------------------------------------------------------

\subsection{Felhasználói felület}
ChakraUI

screenshotok az alkalmazásból
példa kód
téma definiálás
dark mode

%----------------------------------------------------------------------------
