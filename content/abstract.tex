% !TeX spellcheck = hu_HU
\pagenumbering{roman}
\setcounter{page}{1}

\selecthungarian

%----------------------------------------------------------------------------
% Abstract in Hungarian
%----------------------------------------------------------------------------
\chapter*{Kivonat}\addcontentsline{toc}{chapter}{Kivonat}

Rengeteg raktárkezelő alkalmazás érhető el a piacon, azonban a térképes nézet egy ritka funkciónak számít ezekben a rendszerekben. Az erre alkalmas szoftverek rendszerint előredefiniált térképpel dolgoznak.
A szakdolgozatom célja egy olyan raktár kezelő rendszer tervezése és fejlesztése, amely lehetővé teszi az eszközök pozíciójának pontos meghatározását a raktáron belül mindezt dinamikusan. A rendszer segítségével a felhasználók képesek a raktár méretét és elrendezését is meghatározni, így a pontos igényeknek megfelelő alkalmazást kaphatnak.

\vfill
\selectenglish

%----------------------------------------------------------------------------
% Abstract in English
%----------------------------------------------------------------------------
\chapter*{Abstract}\addcontentsline{toc}{chapter}{Abstract}

There is a great deal of warehouse management applications available in the market, however, the map view is a rare feature in these systems. The warehouse management software usually wordks with a predefined maps.
The aim of my dissertation is to design and develop a warehouse management system that allows to determine the exact position of items within the warehouse. With the help of the system, users can also set the size and layout of the warehouse, so they can get the application that suits their exact needs.


\vfill
\selectthesislanguage

\newcounter{romanPage}
\setcounter{romanPage}{\value{page}}
\stepcounter{romanPage}