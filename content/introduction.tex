% !TeX spellcheck = hu_HU
%----------------------------------------------------------------------------
\chapter{\bevezetes}
%----------------------------------------------------------------------------

A félév során a feladatom egy olyan grafikus felülettel ellátott leltár rendszer tervezése és fejlesztése volt, melynek segítségével az alapvető leltári funkciókon felül könnyedén behatárolhatjuk a leltárba vett eszközök pontos helyzetét a raktárunkon belül egy grafikus “térkép” segítségével.

A dolgozatom során első körben egy részletes specifikációt készítettem, melyben taglaltam az alkalmazással szemben támasztott funkcionális és nem-funkcionális követelményeket.
Ezután megterveztem a webalkalmazás felhasználói felületét, egyszerű wireframe-ek segítségével.

A specifikáció és a wireframe-ek elkészítése után kiválasztottam a használni kívánt technológiákat és megterveztem az alkalmazás architekturális felépítését.

Az alkalmazás két fő részből áll, frontend és backend. Utóbbi tartalmazza az üzleti logikát és az adatbázis kommunikációt, míg a frontend az adatok lekéréséért és megjelenítéséért felel, valamint a felhasználói interakciók által eljuttatja a módosításokat a backend részére.
Ezek fejlesztését párhuzamosan végeztem. Minden egyes funkciónak először elkészítettem a backend oldali implementációját, majd hozzáláttam a frontend oldali kód fejlesztésének. Sok esetben szükséges volt a backend módosításra a frontend fejlesztése közben is.

Az alkalmazás tervezésén és fejlesztésén kívül az üzemeltetés előkészítését is elvégeztem, valamint bevezettem olyan megoldásokat, melyek a fejlesztés minőségét segítették elő.

Végül a tesztelésre fektettem a hangsúlyt, amely közben a felmerülő hibák javítását eszközöltem az alkalmazásban.

A dolgozat fejezeteivel próbáltam ezt a sorrendet tartani, hogy az olvasó számára is könnyen követhető legyen a folyamat és az összefüggések.