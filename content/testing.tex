% !TeX spellcheck = hu_HU
%----------------------------------------------------------------------------
\chapter{Tesztelés}
%----------------------------------------------------------------------------

Az alkalmazás működésének validációjában elengedhetetlen lépés a tesztelés.
A tesztelés ezen felül segíti a fejlesztő munkáját is, bármilyen aprónak tűnő változtatás olykor hatással lehet az alkalmazás más részeire is.
Előfordulhat, hogy már meglévő és működő funkciók válnak használhatatlanná új funkciók bevezetése közben.
Ennek a kockázatát megfelelő tesztlefedettséggel minimálisra csökkenthetjük.
%----------------------------------------------------------------------------

\section{Frontend tesztelés}

A frontend teszteléséhez úgy nevezett end–to–end teszteket készítettem.
Az E2E\footnote{end–to–end} tesztek esetén a teszt a felhasználó viselkedését szimulálja.
Egy szimulált böngészőben végzi el a tesztet és tényleges kattintás eseményt vált ki majd figyeli a renderelt képet a tesztnek megfelelő egyezést keresve.

Ezt a Jest\footnote{Jest NPM oldala \url{https://www.npmjs.com/package/jest}} és a Playwright\footnote{Playwright NPM oldala \url{https://www.npmjs.com/package/playwright}} package–ek segítségével valósítottam meg.
A tesztelés során szükség van bizonyos adatbázis adatokra, ezeket környezeti változókba helyeztem.
Jelenleg a tesztek csak lokális környezetben futnak, de a későbbi fejlesztések során így biztosítva lesz, hogy adatbázisból adatok ne kerüljenek harmadik fél kezébe.

Az alábbi példában (\refstruc{lst:e2e}) a bejelentkezés E2E tesztje látható. A bejelentkezési oldalra navigálunk, ahol kitöltjük az email és jelszó mezőket majd elküldjük a formot. A teszt rövid várakozás után ellenőrzi, hogy megtörtént–e az átirányítás és a renderelt oldal tartalmazza–e a Warehouse szöveget egy h2 HTML tag–ben.

\begin{lstlisting}[style=ES6, caption={Bejelentkezés E2E teszt},label={lst:e2e}]
it('should allow users to sign in', async () => {
  await page.goto('http://localhost:3000/auth/login')
  await page.waitForTimeout(1000)
  await page.fill('[name="email"]', process.env.TEST_USERNAME)
  await page.fill('[name="password"]', process.env.TEST_PASSWORD)
  await page.click('[type=submit]')

  await page.waitForTimeout(1000)
  await expect(page.url()).toBe('http://localhost:3000/')
  await expect(page).toHaveText('h2', 'Warehouses')
})
\end{lstlisting}

%----------------------------------------------------------------------------

\section{Backend tesztelés}
A Prisma fejlesztő csapata csupán pár hónappal ezelőtt jelentette be a 2.0–ás verziót.
A folyamatos fejlesztés ellenére is még hiányos az eszközkészlete.
Ennek köszönhetően teszteléshez sem kínál semmilyen megoldást.

A tesztelési környezet kialakításház egy SQLite adatbázist szerettem volna használni, hogy ne a fejlesztés közben használt adatbázist használjam és ne teljen túl sok időbe a tesztek futtatása.
Azonban a Prisma jelenlegi verziója nem támogatja az enumerációt SQLite adatbázis esetén.
Emiatt itt is egy PostgreSQL adatbázissal kellett dolgozom.
A tesztelés előtt egy scripttel automatikusan létrehozom a szükséges adatbázist, majd a tesztek végeztével törlöm azt, így biztosítva, hogy véletlenül se maradjon semmilyen adat az adatbázisba, ami esetleg fals eredményt váltana ki a tesztekből.

A tesztek során a GraphQL végpontot hívtam meg különböző query–kel és mutation–ökkel, majd a választ vizsgálva megbizonyosodtam a helyes működésről.


\begin{lstlisting}[style=ES6, caption=Regisztráció első teszt eset]    
test('successfully register a user', async () => {
  const data = {
    name: 'User Name',
    email: 'test.user@example.org',
    password: 'password',
  }
  const req: any = await request(config.url, register, data)

  expect(req).toHaveProperty('register')
  expect(req.register.user.email).toEqual(data.email)
})
\end{lstlisting}
