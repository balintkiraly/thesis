%----------------------------------------------------------------------------
\chapter{Tesztelés}
%----------------------------------------------------------------------------

Az alkalmazás működésének validációjában elengedhetetlen lépés a tesztelés.
A tesztelés ezen felül segíti a fejlesztő munkáját is, bármilyen aprónak tűnő változtatás olykor hatással lehet az alkalmazás más részeire is.
Előfordulhat, hogy már meglévő és máködő funkciók válnak használhatatlanná új funkciók bevezetése közben.
Ennek a kockázatát megfelelő tesztlefedetséggel minimálisra csökkenthetjük.
%----------------------------------------------------------------------------

\section{Frontend tesztelés}

A frontend teszteléséhez úgy nevezett end-to-end teszteket készítettem.
Az end-to-end tesztek (röviden E2E) esetén a teszt a felhasználó viselkedését szimulálja.
Egy szimulált böngészőben végzi el a tesztet és tényleges kattintás eseményt vált ki és figyeli a kirenderelt képet a tesztnek megfelelő egyezést keresve.

%----------------------------------------------------------------------------

\section{Backend tesztelés}
A Prisma fejlesztú csapata csupűn pár hónappal ezelött jelentette be a 2.0-ás verziót.
A folyamatos fejlesztés ellenére is még hiányos az eszközkészlete.
Ennek köszönhetően teszteléshez sem kinál semmilyen megoldást.

A tesztelési környezet kialakításház egy SQLite adatbázist szerettem volna használni, hogy ne a fejlesztés közben használt adatbázist használjam és ne teljen túl sok időbe a tesztek futtatása.
Azonban a Prisma jelenlegi verziója nem támogatja az enumerációt SQLite adatbázis esetén.
Emiatt itt is egy PostgreSQL adatbázissal kellett dolgozom.
A tesztelés előtt egy scripttel automatikusan létrehozom a szükséges adatbázist, majd a tesztek végeztével törlöm azt, így biztosítva, hogy véltelenül se maradjon semmilyen adat az adatbázisba, ami esetleg fals ereményt váltana ki a tesztekből.

A tesztek során a GraphQL resolver-eket hívtam meg különböző query-kel és mutation-ökkel.

\begin{lstlisting}[style=ES6, caption={Teszt atadbázis migrció}]
await execSync(`${prismaBinary} migrate up --create-db --experimental`, {
    env: {
    ...process.env,
    DATABASE_URL: databaseUrl,
    },
})
\end{lstlisting}

\begin{lstlisting}[style=ES6, caption=Teszt adatbázis törlése]    
const client = new Client({
    connectionString: databaseUrl,
})
await client.connect()
await client.query(`DROP SCHEMA IF EXISTS "test" CASCADE`)
await client.end()
\end{lstlisting}


\begin{lstlisting}[style=ES6, caption=Regisztráció első teszt eset]    
test('successfully register a user', async () => {
  const data = {
    name: 'User Name',
    email: 'test.user@example.org',
    password: 'password',
  }
  const req: any = await request(config.url, register, data)

  expect(req).toHaveProperty('register')
  expect(req.register.user.email).toEqual(data.email)
})
\end{lstlisting}
