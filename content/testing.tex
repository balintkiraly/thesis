%----------------------------------------------------------------------------
\chapter{Tesztelés}
%----------------------------------------------------------------------------

Lorem ipsum

%----------------------------------------------------------------------------
\section{Frontend tesztelés}
%----------------------------------------------------------------------------

Lorem ipsum

%----------------------------------------------------------------------------
\section{Backend tesztelés}
%----------------------------------------------------------------------------

A Prisma fejlesztú csapata csupűn pár hónappal ezelött jelentette be a 2.0-ás verziót.
A folyamatos fejlesztés ellenére is még hiányos az eszközkészlete.
Ennek köszönhetően teszteléshez sem kinál semmilyen megoldást.

A tesztelési környezet kialakításház egy SQLite adatbázist szerettem volna használni, hogy ne a fejlesztés közben használt adatbázist használjam és ne teljen túl sok időbe a tesztek futtatása.
Azonban a Prisma jelenlegi verziója nem támogatja az enumerációt SQLite adatbázis esetén.
Emiatt itt is egy PostgreSQL adatbázissal kellett dolgozom.
A tesztelés előtt egy scripttel automatikusan létrehozom a szükséges adatbázist, majd a tesztek végeztével törlöm azt, így biztosítva, hogy véltelenül se maradjon semmilyen adat az adatbázisba, ami esetleg fals ereményt váltana ki a tesztekből.

A tesztek során a GraphQL resolver-eket hívtam meg különböző query-kel és mutation-ökkel.
